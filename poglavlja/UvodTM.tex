\chapter{Увод}

%Нека општа разматрања о проналажењу одговора на постављено питање..

\section{Опис проблема}
%- генерални опис потребе за ТМ-ом
%- QA као једно од потенцијалних примена 

Проналажење одговора на постављено питање претстваља  свакодневни проблем. Сваким Гугл упитом, покреће се низ алгоритама коју покушавају да одгонетну шта упит заправо представља , које би странице биле релевантне и у ком редоследу. 
Мера "доброг" одговора на постављено питање знатно може да варира у зависности од сврхе система. Интуитивно, тематика одговра и питања може да послужи као добар критеријум одабира квалитетних одговра. Задатак овог рада је испитивање да ли заиста тематика може да помогне у проналажењу адекватног одговра и у коликој мери.

%На пример, промена неких елемената веб странице извршава се стандардном, унапред одређеном процедуром. Уколико би у бази одговора постојали одговори, тј. описи процедура за сваку могућу измену, тада би циљ био пронаћи најбољи одговор за задато питање.  омогућило аутоматско одговарање на питање везана за промену изгледа страница. За овако специфичне и релативно једноставне примене, могуће је користити различите методе претраге. 


\section{Опис решења}


Циљ овог рада је израда прототипа програма који би коришђењем алгоритама за моделовање тема из базе потенцијалних одговора проналазио најбољи одговор за задато питање. Дакле, програм не треба да "осмисли" одговор на задато питање већ само да "препозна" који од могућих одговора највише одговара постављеном питању 

Примена оваквог решења могла би да буде значајна у различитим областима од комерцијалних до научних. На пример, омогућило би се ефикасно аутоматско одговарање на често постављена питања која могу имати различиту формулацију или ефикасно проналажење адекватних научних радова. 

У раду су обрађивани текстови на енглеском језику али због природе модела, развијени програм и резултате могуће је применити и на било који други језик. Поред основног текста питања и одговора, у раду је испитан и утицај додавања синонима на проналажење одговора као и утицај свођења речи на коренске ( склањање глаголских и именских наставака , енг. \textit{stemming})

Као компаративнии модел коришћен је приступ проналажења одговора на основу броја заједничких речи (енг. \textit{WordCount})% при чему се показало да предложено решење има значајно боље резултате.


\section{Терминологија}

Општи преглед заначења термина који су коришћени у раду као што су реч, речник, вокабулар, корипус итд.

\begin{itemize} 
\item Тема : скуп речи које је најбоље карактеришу. На пример тема рачунарсво би предстваљала скуп речи : алгоритам, процесор, кодирање, израчунавање, меморија, рачунар, бит, бајт, лаптоп итд. Важно је приметити да неке речи могу припадати у више тема, као што је нпр. реч израчунавање која може припадати и области математика.
\item Речник или корпус - скуп свих различитих речи које се јављају у неком скупу докумената
\item ....
 \end{itemize}


